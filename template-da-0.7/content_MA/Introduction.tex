\chapter{Introduction}
It is hard to imagine that the history of modern digital computers began in the early 19th century \cite{tanenbaum:mos}.
The English mathematician Charles Babbage (1792-1871) designed his \textit{analytical engine} which was purely mechanical. 
He never got it working though because it was not possible to produce the high precision components that he needed.  Obviously, he did not need an operating system for his analytical machine.
During some of his work, he cooperated with Ada Lovelace who wrote an algorithm for this machine to calculate Bernoulli numbers in 1843 \cite{toole:ateon} and therefore is considered the first programmer.

\begin{figure}[htb]
	\begin{minipage}[ht]{7.5cm}
		\begin{center}
		\includegraphics[scale=1.1]{inputs/pictures_intro/z3}
		\caption[Z3, constructed by Konrad Zuse in 1941]{Z3, constructed by Konrad Zuse in 1941 \cite{uh:kz}} 		\label{fig_z3}
		\end{center}
	\end{minipage}
	\hspace{0.5cm}
	\begin{minipage}[ht]{7.5cm}
		\begin{center}
		\includegraphics[scale=0.9]{inputs/pictures_intro/zynq}
		\caption[Zynq All Programmable SoC from Xilinx (2012)]{Zynq All Programmable SoC from Xilinx (2012) \cite{maxfield:xtzapstg}} \label{fig_zynq}
		\end{center}
	\end{minipage}
\end{figure}  


Not until 100 years later, Zuse, von Neumann and others succeeded to build calculating engines (fig. \ref{fig_z3}). 
Those machines were made of relays and later vacuum tubes, the programming was realized via plugboards and then punched cards.
Still, there was no operating system.
The next computer generation arose with the development of the transistor in the mid-1950s.
For the first time, a program was written down in FORTRAN or assembler, then punched in cards and processed by a computer, all developed by IBM. 
To optimize the process, multiple jobs were recorded on a tape and processed together by the ``ancestor of today's operating system'' \cite{tanenbaum:mos} which was developed concurrently.
With the introduction of \acp{IC}, IBM developed its 360 series.
This series ranged from less powerful and cheaper machines to more powerful and expensive ones.
New was the idea to use the same architecture and instruction set on all machines, what means that the \ac{OS} can run on all machines as well.
Further, the time sharing principle for multiple users was invented later on.
This and the boom of minicomputers influenced the development of the UNIX operating system.
From the early eighties and the introduction of \ac{LSI} circuits, the development of the personal computer dominated and \textit{General Purpose \acp{OS}} like \ac{DOS} and later MicroSoft Windows, Apple's \ac{MAC} and Linux became popular until today. 
Nowadays, very large scale integration circuits allow millions of transistors on one die (fig. \ref{fig_zynq}). 
A new trend is the development of special mobile \acp{OS} for portable devices like smartphones and tablet computers. 

\section{Purpose of Operating Systems}
An operating system is a software program which provides access to the underlying hardware. 
It depends on the hardware setup and the application whether an \ac{OS} is needed.
For simple \ac{I/O} control, e.g. running different color LED lights, no \ac{OS} is required.
Such an application might not even have enough memory to host an \ac{OS}.
\acp{OS} are useful when it comes to managing multiple, more complicated tasks and user \ac{I/O}, for instance in typical general purpose \acp{OS} like MicroSoft Windows.
Its purpose is to manage hardware resources - for example \ac{CPU} time, memory or \ac{I/O} access - as well as system and user processes efficiently. 
The \ac{OS} decides, when a process is allowed to run and access the resources.
Further, \acp{OS} provide synchronization features for task management, for instance semaphores or message passing.
Many \acp{OS} also inherit a set of drivers for different hardware platforms.
This shortens the development process of a new product and therefore the time-to-market.
Depending on the application, there are different kinds of \acp{OS}. 
In General Purpose \acp{OS}, the main goal is to maintain fairness between different users or processes. 
Consequently, every user or process should get an equal time slice of the available \ac{CPU} time or other shared resources like \ac{RAM}. 
\acp{OS} are also used in embedded systems. 
In contrast to General Purpose \acp{OS}, operating systems for embedded devices often have to run under special conditions, e.g. limited memory size or low power consumption. 
Some examples are embedded Linux, FreeRTOS or $\mu$COS.

\subsection{Real-Time Systems}
\acp{RTOS} are a special kind of embedded \ac{OS} which are designed to meet specific timing requirements. 
The main property of an \ac{RTOS} is determinism rather than high-speed performance.  
Further, real-time does not mean that the system is safety critical as this usually implies more safety arrangements, for instance redundancy.
Real-time systems are commonly divided into two classes \cite{stromblad:elfrtoemd}, depending on the consequence caused by missing a deadline:
\begin{itemize}
	\item Hard
	%\item Firm
	\item Soft
\end{itemize}
In hard real-time systems, missing a deadline causes system failure and is not tolerable.  
An example is the engine control system of a car which can be damaged or cause an accident because of delayed signals.
Another example is the release of an airbag in a car. 
It has to be triggered immediately when a crash happens, any delay could cause the loss of lives. 
More applications can be found in medical systems or industry processing control.  
%In firm real-time systems, deadline misses may occur at rare intervals and cause loss of \ac{QoS}, but not a complete system failure.
In soft real-time systems, deadlines can be missed but decrease the \ac{QoS}. 
These kind of systems are usually used for applications with a continuous data flow like multimedia streaming applications or on-line reservation systems.
Missing data packets in these application may cause noise in a video conference but no harm to lives.

\subsection{Choosing the right OS}
The class of a system always depends on its application. To meet the given requirements,
the underlying hardware and - often - the OS have to be chosen appropriately. 
Assuming that the hardware is capable of running more than a light-weight OS, the choice of the operating
system always depends on the real-time requirements of the system. 
Systems which have to provide hard real-time usually must be verified and certified. 
Verification means that each state of the system must be well defined and every change of input leads to a
determined state as well.
Soft real-time does usually not require that kind of verification, but the timings should be reliable. 
In embedded systems, Linux is widely used because it is free of charge, has a large community and therefore a high level of support. 
Moreover, it is comfortable to use as it embeds features like a command terminal, flexible module integration and a large number of available hardware drivers.
Yet, resulting from its structure and complexity, the downside of mainline Linux is still the lack of real-time capability.
\par
To achieve hard real-time performance, usually specific \acp{OS} are used. 
Those can be based on a micro kernel like PikeOS \cite{sysgo:prt} or be designed as light-weight \acp{OS}, for example \textit{$\mu$COS} \cite{micrium:microcos} or \textit{FreeRTOS} \cite{freertos}.
Latter have a very small memory footprint, fast boot times and no background services (deamons) which could unexpectedly disturb the execution of real-time tasks. 
Therefore, the execution time of tasks and interrupt processing is highly predictable.
The jitter (maximum deviation from mean execution time) is very low compared to general purpose \acp{OS}.
Yet, those systems have multiple disadvantages:
In light-weight \acp{OS}, every change in the underlying hardware requires a reconfiguration of the OS.
Furthermore, a high-level communication (e.g. terminal) with the system is not provided and has to be implemented if needed. 
Another solution is one of the hard real-time extensions for Linux like \textit{\ac{RTAI}} \cite{rtai} or \textit{Xenomai} \cite{xenomai}.
Furthermore, companies like ENEA \cite{stromblad:elfrtoemd} make an effort to improve the real-time performance of their proprietary ENEA Linux.  
A hard real-time \ac{OS} not related to Linux is for example PikeOS which was developed by \textit{SYSGO Embedded Systems}.

\subsection{The preferred RTOS}
The operation systems described above are not the optimum choice for developers because either their configuration consumes a lot of time or they cannot be configured to meet the needed real-time requirements.
The optimum system provides all necessary drivers, requires as low configuration and modification as possible and can guarantee the right level of real-time capability. 
Further, it should be free of charge, have a high level of support and run real-time and best-effort tasks\footnote{Best-effort tasks are tasks which do not need to meet any timings.} concurrently.
Most of the systems are either proprietary and therefore not free of charge for development or light-weight systems which barely provide any drivers.
The Linux real-time extension \ac{RTAI} is hard to install and has a complex \ac{API} what makes the use of it unattractive \cite{mitschang:heulfrvr}.
Further, the current \ac{RTAI} version supports only the old Linux kernel version 2.6.38.8 \cite{rtai} whereas the current mainline kernel is already at version 3.12. 
The Xenomai framework uses a microkernel and a hardware abstraction layer to provide hard real-time. 
It already supports the Linux kernel version 3.5.
\par
An open-source alternative which is easy to set up is the Linux PREEMT\_RT patch (further: \textit{LinuxRT})\footnote{From now on, aspects related to both LinuxRT and mainline Linux will be referred to as Linux only.}.
In the past years, a lot of effort was made to improve low-latency and predictability.
It is currently available up to Kernel 3.10.
Moreover, it provides all advantages of mainline Linux which is widely used in embedded systems because it supports many hardware platforms, has a large community and high driver support.
The downside of LinuxRT is that it also inherits some disadvantages of mainline Linux, for instance it is not mathematically provable \cite{clark:itrtlfed}.
Yet, LinuxRT has already been used on computers which have gone into space and it is favored by musicians as ``jitter from mainline Linux would cause a scratching sound'' \cite{clark:itrtlfed}.
LinuxRT sounds very promising for future applications.
\par
Developers want to know before starting the implementation whether an \ac{OS} is suitable for their purposes.
This work aims to provide a guideline which helps to estimate the possibilities and limitations of LinuxRT. 
It should provide information on the overall performance of the system and the real-time capability with maximum latencies.  
To evaluate the results of such a test, it is necessary to compare them to a reference. 
As the main goal is to provide low and deterministic timings, a light-weight \ac{OS} can be used.
The results will not only show the total performance of the \ac{OS} but also bottlenecks which are worth optimizing. 
The guideline should show when is is safe to utilize LinuxRT and when it is inevitable to use a light-weight \ac{OS}.
There are multiple \acp{RTOS} which could be used for a comparison. 
One operating system which became very popular in the last years is FreeRTOS.
It is supported on many platforms, is freely available and already used in different market sectors, for example toys, aircraft navigation or engine control \cite{freertos}. 
Therefore, it is a good candidate to be used as reference.  
The next point to consider is how to benchmark the \acp{OS}.
 
\subsection{Benchmarking RTOSes}
There are many criteria which can be applied to benchmark \acp{OS}, in this case FreeRTOS and LinuxRT.
Typical metrics are latency and jitter - the maximum deviation from mean latency.
For \ac{RTOS}, it is most important to consider features typically used in real-time applications.
Those are mainly task synchronization features, e.g. semaphores, message queues or signals (see section \ref{s_intertask_communication} for details).
Moreover, interrupt latency is crucial because interrupts usually wake up important tasks in \ac{RTOS}.
Another typical application is the periodic invocation of tasks. 
More aspects are the preemption time of tasks and priority inheritance mechanisms to resolve deadlocks.
These two metrics indicate the capability of the system to interrupt a low priority task to run a task with higher priority. 
The performance of all these features depends on the implementation of the \ac{OS}.
Boot time can also be important in real-time systems as it is not acceptable to wait 15 seconds for the engine to run after starting the car.
Further, some systems support memory access supervision by a \ac{MMU} or power saving options.
The use of an \ac{MMU} can be especially important in safety critical systems where not allowed memory accesses can cause a fatal system crash.
Power saving options play an important role in mobile devices.
The right \ac{OS} configuration can significantly increase the run time of such devices.
As \ac{RTOS} are usually used on embedded devices, the memory footprint is also of large interest. 
\par
One challenge is to choose the right criteria from the ones mentioned above. 
Another one is to obtain values which are as accurate and comparable as possible.
Therefore, a way of time measurement has to be found, which can be applied on both systems with the least possible interference on the system.

\section{Related work}
Performance evaluation has already been done on different platforms and with different \acp{OS}. 
The first attempt to develop a real-time benchmark was by Kar and Porter in 1989 by introducing the \textit{Rhealstone Benchmark} \cite{kar:itrb} \cite{kar:artbp}. 
This benchmark is based on six parameters the resulting value is calculated from (for details refer to \ref{ss_rhealstone_benchmark_for_deterministic_application_latency}).
The Rhealstone Performance Number is a weighted mean of all parameters. 
They implemented the benchmark under iRMX.
Recently, Timothy J. Boger applied the Rhealstone Benchmark to FreeRTOS on the ZC702 Evaluation platform \cite{boger:rbofatxzepp}.
His work does not contain measurements on the interrupt latency, though.
\par
Another comparison of operating systems was performed for a set of \ac{OS} suitable for small microcontrollers \cite{Anh:sapeortosfsm}. 
In this paper, several \ac{RTOS} are introduced, but only four of them are chosen to be investigated in detail.
The choice is based on available support, documentation, scheduling type and more. 
For the \acp{RTOS}, algorithms on how to compare task switching, message passing, semaphore passing and memory allocation are presented.  
In \cite{gokhan:cstamfcoxamom}, memory footprint and context switching is compared between $\mu$COS and XilKernel\footnote{XilKernel \cite{xilinx:xilkernel} is an embedded \ac{OS} developed by Xilinx.} on the softcore processor Microblaze.
Three different \acp{OS} - Xenomai, LinuxRT and eCos \cite{ecos} - are compared in \cite{Marieska:opokbaertosbaa}.
The performance metrics are similar to the Rhealstone benchmark not including the interrupt latency.
Depending on the application, eCos and LinuxRT perform better than Xenomai.
The authors of \cite{cereia:peoaemulatrp} and \cite{betz:eeotlrpfrta} evaluate the real-time performance of LinuxRT compared to standard Linux.
In both works, a sample program is run solely, with CPU and I/O load. 
Whereas in \cite{cereia:peoaemulatrp} the focus is mainly on the effect of real-time priorities, in \cite{betz:eeotlrpfrta} different period lengths and multi-processor effects are also investigated. 
The results show a significant reduction in jitter when using LinuxRT.
\par
Although some investigation has already been done on LinuxRT, only few works deal with interrupt processing of the \ac{OS}.
The complete Rhealstone benchmark has neither been implemented for LinuxRT nor for FreeRTOS.
Further, none of the works tries to model the effect on application latency using the results they obtained.
Generally, there is little work available on FreeRTOS.

\section{Contribution}
This work presents deeper investigation on LinuxRT and FreeRTOS based on the Rhealstone benchmark.
The goal is to provide a guideline on the choice of an operating system for specific use cases.
This is accomplished by
	\begin{itemize}
		\item Introducing a model for application latency including \ac{OS} depending aspects
		\item Filling the parameters of this model for LinuxRT and FreeRTOS by implementing tests based on the Rhealstone benchmark proposed by Kar and Porter \cite{kar:itrb} \cite{kar:artbp} 
		\item Utilizing the model to calculate the latency for a synthetic application 
		\item Deriving application design guidelines for Linux in general based on the results
	\end{itemize} 
The Rhealstone benchmark metrics are extended by the boot time and a detailed investigation on interrupt latency for LinuxRT. 
It is performed under load and in idle condition and further compared to the mainline kernel.
The benchmark is implemented for both \acp{OS} on the ZC702 evaluation board which inherits a Zynq chip (for details on the platform refer to section \ref{s_test_environments_and_tools}). 
Instructions on how to implement the benchmarks are provided, so the tests can be performed for different \acp{OS} as well.
Alternatively, the given sources can be used to benchmark different platforms after little modification.
\par
The results show that the LinuxRT kernel performs well in terms of interrupt latency for soft real-time requirements, even with load.
Moreover, the benchmark results expose weaknesses in the Linux POSIX compliant \ac{API} which can be considered during application design. 
%FreeRTOS is one of the most important light-weight \ac{RTOS} nowadays.
%As there is a large interest in using more comfortable \acp{OS} like Linux, the goal is to clearly define the limits of LinuxRT.
%The optimum result is a LinuxRT configuration with comparable benchmark results as FreeRTOS. 
\par
The remainder of the work is organized as follows. In chapter \ref{ch_background}, a detailed background on operating systems and an application latency model are provided. 
Chapter \ref{ch_measurements} describes the performed algorithms and measurements in detail. 
Further, in chapter \ref{ch_results} the results of the measurements are presented. 
In chapter \ref{ch_estimation_of_application_run_time}, the application latency model and the results from chapter \ref{ch_results} are used to estimate the run time of two theoretic applications.  
The last chapter summarizes the contribution of this work and gives an outlook on future work.