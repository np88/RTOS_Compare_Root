\chapter{Concluding Remarks}\label{ch_conclusion}
Concluding, the first part of this chapter gives a brief summary of the development process[Wort] in this work.
The second part introduces possibilities of future work and extensions.

\section{Summary and Reflection}   
In the following, the main parts of this thesis are summarized:

\begin{itemize}
	\item 
	The first chapter gives an introduction to the topic. 
	It explains the timing problems in real-time systems and the role and challenges of \acp{RTOS}.
	Moreover, it outlines why the \acp{OS} LinuxRT and FreeRTOS were chosen for the comparison.
	\item
	In the second chapter, a detailed background on operation systems is given with focus on FreeRTOS and LinuxRT.
	Based on that, sources of delays caused by hard ware and the \ac{OS} are analyzed.
	Finally, it is discussed which delays occur in the \acp{OS} under test and how they can be minimized or completely eliminated.
	\item
	Derived from the analysis in chapter two, a set of benchmarks to quantify the operation systems is defined in chapter three.
	These are mainly based on the real-time Rhealstone benchmark proposed by Kar.
	Some extensions to measure boot time and detailed interrupt handling were made.
	Further, the underlying test hardware platform is introduced.
	\item
	The fourth chapter contains the results of the test which were described in chapter three.
	Further, a comparison and interpretation of these results is given.
	One important result is the maximum interrupt latency under network load.
	Moreover, the complex implementation of the Linux POSIX \ac{API} is clearly shown in the results.
	This enables concrete optimization of sources which have strong impact on the system behavior.	
	The results obtained give the possibility to compare different synchronization methods, tune the performance of an application and estimate whether an \ac{OS} is suitable for a specific task or not.
\end{itemize}

The goal of this work was to make a comparison between the two operation systems LinuxRT and FreeRTOS. 
The difficulty of this task is the big difference between those two operation system and to justify such a comparison.
FreeRTOS is a light-weight system with a low memory footprint which is certifiable, very fast and predictive. 
Jitter from the \ac{OS} is barely available. 
The drawback is that any change in the applications includes recompilation and reprogramming of the hardware.
Moreover, every driver and its handling has to be implemented manually if needed.
LinuxRT supports lots of drivers and is convenient for application design, but the complexity of the system has a high impact on its real-time capability.
This work uses the Rhealstone method to create a benchmark suite and to obtain the possibilities and limitations of the two \acp{OS}.

