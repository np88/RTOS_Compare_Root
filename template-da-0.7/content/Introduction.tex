\chapter{Einleitung}
\section{Realzeitbetriebssysteme}
	\begin{itemize}
		\item Was sind RTOS und was macht sie aus?
		\item Warum verwendet man sie anstatt herk�mmlicher Betriebssysteme oder dedizierter Hardware?
		\item Verschiedene Arten von RTOS und ihre Vor-/Nachteile (Wof�r eignen sich bestimmte Systeme besonders gut, grober �berblick �ber vorhandenes)
		\item Beschreiben der Zielhardware/Randbedingungen
		\item Auskristallisieren, warum in der Arbeit gerade Linux RT Patch und FreeRTOS verwendet werden (evtl noch andere, z.B. MicroCOS, Xenomai)	
	\end{itemize}
	
\section{Welche Eigenschaften werden verglichen?}
\section{Relevante Betriebssysteme}
\subsection{Linux}
\subsection{FreeRTOS}
	\begin{itemize}
		\item Tasks k�nnen die gleiche Priorit�t haben
		\item Tick rate bestimmt Zeitaufl�sung $ \rightarrow $ je �fter es aufgerufen wird, desto mehr Zeit wird f�r das Betriebssystem aufgewendet (Task switches werden dann ausgef�hrt)
		\item Normalerweise sollten ISR so kurz wie m�glich sein. Deswegen wird ein hoch priorisierter Task aus der ISR aufgerufen, der eine Priorit�t gr��er oder gleich dem System-Interrupt hat und wird dadurch nicht durch das System unterbrochen
		\item A mutex, binary semaphore and a counting semaphore is using the existing queue mechanism.
		\item Mutexes include priority inheritance mechanism, binary semaphores do not. This makes binary semaphores the better choice for implementing synchronisation (between tasks or between tasks and an interrupt), and mutexes the better choice for implementing simple mutual exclusion.  
		\item Rekursiver Mutex: Ein Mutex kann mehrfach gelockt werden, muss aber auch mehrfach wieder entlockt werden.
	\end{itemize}

