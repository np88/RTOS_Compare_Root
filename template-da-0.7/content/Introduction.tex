\chapter{Einleitung}
\section{Realzeitbetriebssysteme}

		
\subsubsection{Was sind RTOS und was macht sie aus?}
Im Allgemeinen ist ein Betriebssystem daf�r verantwortlich, die Hardwareressourcen von einem System zu verwalten sowie f�r die Verwaltung von Benutzer- und Systemprozessen. Es gibt verschiedene Arten von Betriebssystemen. Im Alltag werden so genannte General Purpose OS verwendet, zum Beispiel Microsoft Windows oder MAC OS. Diese versuchen, die Ressourcen m�glichst fair einzusetzen und die Antwortzeiten gegen�ber dem Benutzer so gering wie m�glich zu halten. In einem Realzeitbetriebssystem liegt der Fokus auf dem Garantieren von zeitlicher Determinierung von kritischen Anwendungen. Bestimmte Prozesse m�ssen in einer maximal festgelegten Zeit abgeschlossen sein, da das System sonst instabil werden k�nnte. Das ganze kann man an einem Beispiel greifbar machen:
\\Bei einem General Purpose OS kann es mehrere Sekunden dauern, bis sich der Internet Browser ge�ffnet hat. W�rde es beispielsweise in einem Auto mehrere Sekunden dauern, bis sich der Airbag ge�ffnet hat, kann ein Mensch bereits verletzt worden sein. Dieses ist eine zeitkritische Anwendungen, die eine deterministische Antwortzeit garantieren muss. Bei Echtzeit unterscheidet man zwischen harter und weicher Echtzeit. Wenn bei harter Echtzeit die Deadlines verletzt werden, dann f�hrt das zu einem Versagen des Systems und hat Folgen wie das Verletzten von Menschenleben. Bei weicher Echtzeit haben Verletzungen der Deadlines gegebenfalls einen Einfluss auf die QOS, f�hren aber �blicherweise nicht zum Absturz des Systems oder bergen Gefahr f�r beteiligte Personen. 

\subsubsection{Warum verwendet man sie anstatt herk�mmlicher Betriebssysteme oder dedizierter Hardware?}
\subsubsection{Verschiedene Arten von RTOS und ihre Vor-/Nachteile} 
Wof�r eignen sich bestimmte Systeme besonders gut, grober �berblick �ber vorhandenes

\subsubsection{Beschreiben der Zielhardware/Randbedingungen}
\subsubsection{Warum FreeRTOS und Linux}
Auskristallisieren, warum in der Arbeit gerade Linux RT Patch und FreeRTOS verwendet werden (evtl noch andere, z.B. MicroCOS, Xenomai)	
	
\section{Welche Eigenschaften werden verglichen?}
\section{Relevante Betriebssysteme}
\subsection{Linux}
\subsubsection{Prozesse}
Prozesse in Linux sind als doppelt verkettet Liste implementiert. Die Prozesse im State \textit{Running} haben eine Liste pro Priorit�t. Prozesse in States wie \textit{Task\_Stopped}, \textit{Exit\_Zombie} oder \textit{Exit\_Dead} werden nicht in speziellen Listen gespeichert, weil der Zugriff meistens �ber PID oder Kind Prozesse stattfindet. Tasks, die interruptable oder nicht interruptable sind, m�ssen feiner gruppiert werden, um den Anforderungen der entsprechenden Funktionalit�t zu gen�gen. Zum Beispiel werden wartende (schlafende) Prozesse in einer Waiting Queue eingereiht. Je nach dem, ob ein Ereignis eintrifft, oder eine Ressource frei wird, werden entsprechende Prozesse wieder aufgeweckt. Context Switches werden nur im Kernel Modus vollzogen. Der Switch wird �ber Software gesteuert, nicht �ber Hardware. 
\subsubsection{Scheduling Policy}
The scheduler stores the records about the planned tasks in a red-black tree, using the spent processor time as a key. This allows it to pick efficiently the process that has used the least amount of time (it is stored in the leftmost node of the tree). The entry of the picked process is then removed from the tree, the spent execution time is updated and the entry is then returned to the tree where it normally takes some other location. The new leftmost node is then picked from the tree, repeating the iteration.\\\\
If the task spends a lot of its time sleeping, then its spent time value is low and it automatically gets the priority boost when it finally needs it. Hence such tasks do not get less processor time than the tasks that are constantly running.
\\\\CFS is an implementation of a well-studied, classic scheduling algorithm called weighted fair queuing.
Summing up, CFS works like this: it runs a task a bit, and when the task
schedules (or a scheduler tick happens) the task's CPU usage is "accounted
for": the (small) time it just spent using the physical CPU is added to
p->se.vruntime.  Once p->se.vruntime gets high enough so that another task
becomes the "leftmost task" of the time-ordered rbtree it maintains (plus a
small amount of "granularity" distance relative to the leftmost task so that we
do not over-schedule tasks and trash the cache), then the new leftmost task is
picked and the current task is preempted.
\\\\CFS doesn't use priorities directly but instead uses them as a decay factor for the time a task is permitted to execute. Lower-priority tasks have higher factors of decay, where higher-priority tasks have lower factors of delay. This means that the time a task is permitted to execute dissipates more quickly for a lower-priority task than for a higher-priority task. That's an elegant solution to avoid maintaining run queues per priority. 

\subsubsection{Interrupts}
Regardless of the kind of circuit that caused the interrupt, all I/O interrupt handlers perform the
same four basic actions:
\begin{enumerate}
	\item Save the IRQ value and the register's contents on the Kernel Mode stack.
	\item Send an acknowledgment to the PIC that is servicing the IRQ line, thus allowing it to issue further interrupts.
	\item Execute the interrupt service routines (ISRs) associated with all the devices that share
the IRQ.
	\item Terminate by jumping to the \textit{ret\_from\_intr} address.
\end{enumerate}

\subsection{RT Linux}
Um Linux RT f�higer zu machen, werden IRQs anders behandelt als im normalen Linux. Im normalen Linux unterbrechen sie einfach den laufenden Betrieb des Betriebssystems. Mit dem RT-Patch werden die ISRs in eigene Prozesse mit Priorit�ten umgewandelt, sodass RT-Prozesse eine h�here Priorit�t haben k�nnen als IRQs und damit weniger anf�llig f�r Unterbrechungen sind. IRQs werden im RT-Patch mit der FIFO\_SCHED Policy geschedulet und haben eine Priorit�t von 50.
\\\\In PREEMPT\_RT, normal spinlocks (spinlock\_t and rwlock\_t) are preemptible, as are RCU read-side critical sections. Es gibt aber eine zus�tzliche Art von Spin Locks, die das traditionelle Verhalten erlaubt.

\subsection{FreeRTOS}
	\begin{itemize}
		\item Tasks k�nnen die gleiche Priorit�t haben
		\item Tick rate bestimmt Zeitaufl�sung $ \rightarrow $ je �fter es aufgerufen wird, desto mehr Zeit wird f�r das Betriebssystem aufgewendet (Task switches werden dann ausgef�hrt)
		\item Normalerweise sollten ISR so kurz wie m�glich sein. Deswegen wird ein hoch priorisierter Task aus der ISR aufgerufen, der eine Priorit�t gr��er oder gleich dem System-Interrupt hat und wird dadurch nicht durch das System unterbrochen
		\item A mutex, recursive mutex, binary semaphore and a counting semaphore is using the existing queue mechanism.
		\item Mutexes include priority inheritance mechanism, binary semaphores do not. This makes binary semaphores the better choice for implementing synchronisation (between tasks or between tasks and an interrupt), and mutexes the better choice for implementing simple mutual exclusion.  
		\item Rekursiver Mutex: Ein Mutex kann mehrfach gelockt werden, muss aber auch mehrfach wieder entlockt werden.
	\end{itemize}

