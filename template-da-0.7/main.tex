%LaTeX
\typeout{$Id$}
%
%%%%%%%%%%%%%%%%%%%%%%%%%%%%%%%%%%%%%%%%%%%%%%%%%%%%%%%%%%%%%%%%%%%%%%
%&delatex
\documentclass[%
	12pt,		% 12pt font size
	headinclude,	% 
	%openany,	% Kapitel k�nnen auch auf der linken Seite anfangen.
	a4paper,	% Papierformat
	numbers=noenddot, % entgegen Duden Nr. 21, S. 21 R4 als CI festgelegt
	changebar,      % �nderungsbalken mit \begin/\end{changebar}
	listof=totoc,	% Was alles in der Table of Contents stehen soll
	bibliography=totoc,
	index=totoc,
	version=first,
	%draft,		% mark overfull hboxes with a thick rule, don't include
	%		% (EPS) files, begin bibliography on a new page
	]{diplomarbeit}
	% noheadsepline

\areaset[20mm]{160mm}{240mm}	% Binderand, Breite und Höhe nutzbarer Bereich

%\usepackage{bibgerm}		% F�r geralpha.bst. MUSS vor babel stehen
%\usepackage{pslatex}
% For english, uncomment the following two lines and comment out the
% two lines below that. Don't forget to 'make clean' when switching 
% between languages.
%\usepackage[german,english]{babel} % Multilingual text, order is important !
%\selectlanguage{english}        % Select default language
\usepackage[ngerman]{babel} % Mehrsprachiger Text, Reihenfolge wichtig!
\usepackage[latin1]{inputenc}	% Latin1 input encoding (umlauts)
				   % Die zuletzt genannte Sprache wird verwendet!
\selectlanguage{ngerman}	% W�hle Standardsprache ...
\usepackage{version}		% handle comments and different versions
				% \begin/end{comment},
				% \include/excludeversion{foo}
\usepackage{longtable}		% \begin{longtable} ... see LaTeX Companion
\usepackage{parskip}		% seperate paragraphs by skip (blank line)
%\usepackage{amsfonts,amsmath,amssymb,array} % choose if needed
%\usepackage{ae,aecompl}	% Benutze 'Almost European' Zeichensatz         
%\usepackage{type1cm}	% Computer Modern type 1 fonts
\usepackage{lmodern}	% Latin Modern nehmen, da mehr Glyphen vorhanden 
\usepackage[T1]{fontenc}	% Verwende T1-Kodierung f�r Zeichens�tze        


\usepackage{changebar}
\usepackage{makeidx}\makeindex

% --------------------------------------------------------------------
% Typ der Arbeit
% --------------------------------------------------------------------
% \arbeitsart{Techreport}
% \techreportident{TR-XXXXXYY}
% \arbeitsart{Studienarbeit}
% \arbeitsart{IDP}
% \arbeitsart{Praktikumsbericht}
% \arbeitsart{Bachelor}
% \arbeitsart{Diplomarbeit}
\arbeitsart{Master's Thesis}

% --------------------------------------------------------------------
% Titel, Autor und sonstige Titelblatt--Angaben:
% \betreuer{}, \band{} und \eingereicht{} setzen die
% Style--Option `datitlepage' voraus!
% --------------------------------------------------------------------
\title{Das ist ein etwas l�ngerer Titel\\
        dieses leeren Diplomarbeitsger�sts}
\author{Nadja Peters}
\authorsaddr{Arcisstra�e 21\\
             80333 M�nchen}

\betreuer{Dipl.--Ing. Philipp Kindt}
% --------------------------------------------------------------------
% bei externer Diplomarbeit
% --------------------------------------------------------------------
\ausgefuehrtam{bei Siemens}
%%% oder
%\ausgefuehrtam{%
%  am Institut f�r Werkzeugmaschinen und Betriebswissenschaften\\
%   Technische Universit�t M�nchen}
%%%_optional: \band{Band1}
%%%_optional: \eingereicht{im September 2004}

% --------------------------------------------------------------------
% eigene Definitionen f�r die Praeambel
% --------------------------------------------------------------------
% commonly used shorthands:
\newcommand{\qq}[1]{\glqq#1\grqq}	% Vereinfachung
\newcommand{\zb}{z.\,B.\ }              % Vereinfachung
\newcommand{\zB}{z.\,B.\ }              % manche von uns schreiben lieber zB
\newcommand{\dht}{d.\,h.\ }             % Vereinfachung
\newcommand{\ua}{u.\,a.\ }              % Vereinfachung
\newcommand{\bzw}{bzw.\ }               % Vereinfachung
\newcommand{\ggf}{ggf.\ }               % Vereinfachung
\newcommand{\etc}{etc.\ }               % Vereinfachung
\newcommand{\evtl}{evtl.\ }             % Vereinfachung
\newcommand{\bzgl}{bzgl.\ }             % Vereinfachung
\newcommand{\iA}{i.\,A.\ }              % Vereinfachung
\newcommand{\sog}{sog.\ }               % Vereinfachung
\newcommand{\ca}{ca.\,}
\newcommand{\vgl}{vgl.\,}
\newcommand{\usw}{usw.\ }
\newcommand{\va}{v.\,a.\ }
\newcommand{\idR}{i.\,d.\,R.\ }

% commonly used shorthands:
\newcommand{\Cf}{Cf.\ }
\newcommand{\cf}{cf.\ }
\newcommand{\Eg}{E.g.\ }
\newcommand{\eg}{e.g.\ }
\newcommand{\Ie}{I.e.\ }
\newcommand{\ie}{i.e.\ }
\newcommand{\vs}{vs.\ }

% shorthands for units
%\newcommand{\Bar}[1]{\overline{#1}} %conflicts with ams*
\newcommand{\Unit}[1]{\mbox{\,#1}}

\newcommand{\bit}{\Unit{bit}}
\newcommand{\kB}{\Unit{kB}}
\newcommand{\MB}{\Unit{MB}}
\newcommand{\Mb}{\Unit{MBit}}
\newcommand{\GB}{\Unit{GB}}

\newcommand{\Hz}{\Unit{Hz}}
\newcommand{\kHz}{\Unit{kHz}}
\newcommand{\MHz}{\Unit{MHz}}
\newcommand{\GHz}{\Unit{GHz}}

\newcommand{\ms}{\Unit{ms}}
\newcommand{\ns}{\Unit{ns}}
\newcommand{\us}{\Unit{$\mu$s}}

\newcommand{\prs}{\Unit{s$^{-1}$}}
\newcommand{\MBs}{\MB\prs}
\newcommand{\Mbs}{\Mb\,\prs}
\newcommand{\GBs}{\GB\prs}
\newcommand{\Bins}{\Unit{Bildern}~\prs}
\newcommand{\Bis}{\Unit{Bilder}~\prs}

% specific shorthands

\newcommand{\pla}{Plattform A}
\newcommand{\plb}{Plattform B}
\newcommand{\plc}{Plattform C}



% --------------------------------------------------------------------
% PDF specials
% --------------------------------------------------------------------
	\pdfcompresslevel=9
        \RequirePackage[pdftex,		
		      %% f�r die DRUCK-Ausgabe bitte folgende
		      %% Einstellungen verwenden
		       pdfpagelabels=true,
                       bookmarks=true,
                       pagebackref=true,       % Rücklinks im Lit.-verz.
                       linktocpage=true,
                       plainpages=false,
                       hyperfootnotes=false,   % sonst doppelt definiert
                       breaklinks=true,
                       colorlinks=true,
                       linkcolor=black,
                       urlcolor=black,
                       citecolor=black]{hyperref} 
		%% f�r die .pdf-Datei-Ausgabe bitte diese
		%% Einstellungen verwenden und obige auskommentieren! 
%		       pdfpagelabels=true,
%		       bookmarks=true,
%		       pagebackref=true,	% R�cklinks im Lit.-verz.
%		       linktocpage=true,
%		       plainpages=false,
%		       hyperfootnotes=false,	% sonst doppelt definiert
%		       breaklinks=true,
%		       colorlinks=true,
%		       linkcolor=blue]{hyperref}
	\makeatletter
	\hypersetup{
		pdftitle={\@title},
		pdfauthor={\@author},
		pdfsubject={Diplomarbeit},
		pdfkeywords={keyword1,keyword2},
	}
	\makeatother

% wenn nur einzelne Kapitel gedruckt werden sollen...
%
%\includeonly{meineergebnisse}

% --------------------------------------------------------------------
%%% Hier beginnt das eigentliche LaTeX--Dokument
% --------------------------------------------------------------------
\begin{document}
% --------------------------------------------------------------------
% Titelseite
% --------------------------------------------------------------------
\renewcommand{\thepage}{t\arabic{page}}
\maketitle

% --------------------------------------------------------------------
% Dank 
% --------------------------------------------------------------------
 \pagestyle{empty}
\cleardoublepage 
\setcounter{page}{3}	% Laut VDI Richtlinien
\renewcommand{\thepage}{\roman{page}}
\begin{danksagung}
  Vielen Dank \dots

  M�nchen, im Monat Jahr
\end{danksagung}

% Hier k�nnte eine Widmung stehen!
\begin{Widmung}
% oder auch nicht. ;)
\end{Widmung}

% --------------------------------------------------------------------
% Inhaltsverzeichnis, Abbildungsverzeichnis, Tabellenverzeichnis
% --------------------------------------------------------------------
\clearpage
\pagestyle{headings}
% Inhaltsverzeichnis soll (nur) in den PDF Bookmarks erscheinen
\pdfbookmark[1]{\contentsname}{toc}
\tableofcontents

\listoffigures
\listoftables
% --------------------------------------------------------------------
% Liste der verwendeten Symbole
% --------------------------------------------------------------------
\phantomsection
\begin{listofsymbols}
\begin{tabbing}
sp	\= symbol-space---- \= \kill \+ \\

RCS	\> Lehrstuhl f�r Realzeit-Computersysteme \\
\end{tabbing}
\end{listofsymbols}

% --------------------------------------------------------------------
% Abstract (Kurzfassung)
% In Deutsch und Englisch zu verfassen laut Promotionsordnung, zumindest
% bei elektronischer Veröffentlichung.
% --------------------------------------------------------------------
%\pagestyle{empty}
\cleardoublepage 
\begin{abstract}
Die Kurzfassung \dots
\end{abstract}

% Man muss erzwingen, dass die Seite mit Nummer 1 rechts ist, sonst
% kommt latex aus dem Tritt, weil ungerade Seiten immer rechts sein
% müssen!
\newpage{\pagestyle{plain}\cleardoublepage}
\rmfamily	% ab jetzt mit Serifen, weil besser lesbar
% Die Seitennumerierung wird hier mit "1" begonnen
\renewcommand{\thepage}{\arabic{page}}
\setcounter{page}{1}


% --------------------------------------------------------------------
% Hier beginnt der eigentliche Text
% --------------------------------------------------------------------
% Die Identifikation wesentlicher eigener Beiträge muß schon im
% Inhaltsverzeichnis m�glich sein.
\chapter{Einleitung}
\section{Realzeitbetriebssysteme}
	\begin{itemize}
		\item Was sind RTOS und was macht sie aus?
		\item Warum verwendet man sie anstatt herk�mmlicher Betriebssysteme oder dedizierter Hardware?
		\item Verschiedene Arten von RTOS und ihre Vor-/Nachteile (Wof�r eignen sich bestimmte Systeme besonders gut, grober �berblick �ber vorhandenes)
		\item Beschreiben der Zielhardware/Randbedingungen
		\item Auskristallisieren, warum in der Arbeit gerade Linux RT Patch und FreeRTOS verwendet werden (evtl noch andere, z.B. MicroCOS, Xenomai)	
	\end{itemize}
	
\section{Welche Eigenschaften werden verglichen?}
\section{Relevante Betriebssysteme}
\subsection{Linux}
\subsection{FreeRTOS}
	\begin{itemize}
		\item Tasks k�nnen die gleiche Priorit�t haben
		\item Tick rate bestimmt Zeitaufl�sung $ \rightarrow $ je �fter es aufgerufen wird, desto mehr Zeit wird f�r das Betriebssystem aufgewendet (Task switches werden dann ausgef�hrt)
		\item Normalerweise sollten ISR so kurz wie m�glich sein. Deswegen wird ein hoch priorisierter Task aus der ISR aufgerufen, der eine Priorit�t gr��er oder gleich dem System-Interrupt hat und wird dadurch nicht durch das System unterbrochen
	\end{itemize}


\chapter{Benchmarking}
Vergleichskriterien f�r Betriebssysteme:
\begin{itemize}
	\item Performanz
	\item Sicherheit
	\item Ressourcen-/Speicherverbrauch
	\item Speicherverwaltung
\end{itemize}

\section{Performanz}
\begin{enumerate}
	\item Latenzzeiten/Jitter
		\begin{enumerate}
			\item Interrupt durch Taster $ \rightarrow $ Aufblinken von LED (die Zeit, die dasAufblinken ben�tigt, kann gemessen und abgezogen werden, so dass nur die Zeit vom Dr�cken des Tasters bis zum Ausf�hren der ISR bleibt)
			\item Andere Interruptquellen? ( $ \rightarrow $ z.B CAN, Ethernet, SPI, ...)
			\item Verschiedene Taktzeiten von FreeRTOS
		\end{enumerate}
	\item Durchsatz an Daten
		\begin{enumerate}
			\item Ethernet
			\item CAN
			\item SPI
		\end{enumerate}	
	\item Bootzeit
		\begin{enumerate}
			\item Was hat Einwirkungen auf die Bootzeit?
			\item Indikatoren $ \rightarrow $ Wann ist das System hochgefahren?
			\item Bestimmtes Programm wird gestartet $ \rightarrow $ z.B. Aufleuchten von LED
			\item Bestimmte Programme k�nnen den Bootvorgang aufzeichen (Bootchart)  
		\end{enumerate}			
\end{enumerate}

\subsection{Latenzzeiten von Interrupts}
F�r die Latenzzeit von Interrupts soll ein Interrupt von einer externen Quelle ausgel�st werden und dann wird gemessen, wann die Interruptserviceroutine betreten wird. Konkret wird periodisch ein GPIO-Interrupt durch einen Signalgenerator in Hardware ausgel�st. Die GPIO wird �ber das EMIO-Interface angebunden. In der dazugeh�rigen ISR wird eine LED angeschaltet. Es wird die Zeit zwischen dem Setzen des Interruptsignals und aufblinken der LED gemessen. Von dieser Zeit muss abgezogen werden, wie lange das Anschalten der LED und die Zeitmessung an sich dauert. Grunds�tzlich wird jede Messung 1024 Mal durchgef�hrt. F�r die Zusatzmessungen sind die Durchschnittszeiten interessant. F�r die Hauptmessung ist zus�tzlich der Worst-Case-Fall zu beachten.
 
\subsubsection{FreeRTOS}
Bei FreeRTOS werden die Interrupts unabh�ngig vom Betriebssystem verwaltet. Der EMIO-GPIO-Interrupt hat nach dem System-Timer die h�chste Priorit�t. 
\\\\Axi-Timer: 50 MHz

Ergebnis: 

\begin{table*}[htb]
	\centering
		\begin{tabular}{|l|l|}
			\hline
			 Anz. Messungen &  19100 \\	
			\hline
			 Durchschnittswert & 754,8ns \\
			\hline
			 Standardabweichung &  8,5977ns (19.31 ns) \\
			\hline
			 Minimalwert & 740ns \\
			\hline	
			 Maximalwert &  780ns \\
			\hline
			 Durchschnittswert LED-Anschalten &  700-740ns (130 ns)\\
			\hline
			 Durchschnittswert Timer-Overhead &  520ns (9ns) \\
			\hline				
			LED-Overhead bei ISR-Messung & 220ns (121 ns)\\
			\hline		
			Durchschnitt - Overhead & 633,8ns \\
			\hline	
			Durchschnitt - Overhead (Taktzyklen) & 422 Zyklen\\
			\hline								
		\end{tabular}
		\label{theap1}
\end{table*}

mit Task:
\begin{table*}[htb]
	\centering
		\begin{tabular}{|l|l|}
			\hline
			 Anz. Messungen &  19100 \\	
			\hline
			 Durchschnittswert & 754,8ns \\
			\hline
			 Standardabweichung &  8,5977ns (19.31 ns) \\
			\hline
			 Minimalwert & 719ns \\
			\hline	
			 Maximalwert & 840ns \\
			\hline
			 Durchschnittswert LED-Anschalten &  700-740ns (130 ns)\\
			\hline
			 Durchschnittswert Timer-Overhead &  520ns (9ns) \\
			\hline				
			LED-Overhead bei ISR-Messung & 220ns (121 ns)\\
			\hline		
			Durchschnitt - Overhead & 633,8ns \\
			\hline	
			Durchschnitt - Overhead (Taktzyklen) & 422 Zyklen\\
			\hline								
		\end{tabular}
		\label{theap1}
\end{table*}

\subsubsection{LinuxRT}
Messung mit wmb():
Ergebnis: 
\begin{table*}[htb]
	\centering
		\begin{tabular}{|l|l|}
			\hline
			 Anz. Messungen &  19160 \\	
			\hline
			 Durchschnittswert & 13,569us \\
			\hline
			 Standardabweichung &  1,4412us \\
			\hline
			 Minimalwert & 8us \\
			\hline	
			 Maximalwert &  28,8us \\
			\hline
			 Durchschnittswert LED-Anschalten &   \\
			\hline
			 Durchschnittswert Timer-Overhead &   \\
			\hline				
			LED-Overhead bei ISR-Messung &  \\
			\hline		
			Durchschnitt - Overhead & s \\
			\hline	
			Durchschnitt - Overhead (Taktzyklen) & \\
			\hline								
		\end{tabular}
		\label{theap1}
\end{table*}

Messung ohne wmb():
Ergebnis: 
\begin{table*}[htb]
	\centering
		\begin{tabular}{|l|l|}
			\hline
			 Anz. Messungen &  19130 \\	
			\hline
			 Durchschnittswert & 13,406us \\
			\hline
			 Standardabweichung &  1,3419us \\
			\hline
			 Minimalwert & 8,6us \\
			\hline	
			 Maximalwert &  32us \\
			\hline
			 Durchschnittswert LED-Anschalten &   \\
			\hline
			 Durchschnittswert Timer-Overhead &   \\
			\hline				
			LED-Overhead bei ISR-Messung &  \\
			\hline		
			Durchschnitt - Overhead & s \\
			\hline	
			Durchschnitt - Overhead (Taktzyklen) & \\
			\hline								
		\end{tabular}
		\label{theap1}
\end{table*}

Messung For-Loops in Zyklen:
\begin{table*}[htb]
	\centering
		\begin{tabular}{|l|l|l|}
			\hline	
			For-Loop 0 & Runden 16 &  180  \\
			\hline	
			For-Loop 1 & Runden 32 &  304  \\
			\hline	
			For-Loop 2 & Runden 64 &  533 \\
			\hline	
			For-Loop 3 & Runden 128 &  1045  \\
			\hline	
			For-Loop 4 & Runden 264 &  2133  \\
			\hline	
			For-Loop 5 & Runden 512 &  4117  \\
			\hline	
			For-Loop 6 & Runden 1024 &  8213  \\
			\hline	
			For-Loop 7 & Runden 2048 &  16405  \\
			\hline	
			For-Loop 8 & Runden 4096 &  32789  \\
			\hline	
			For-Loop 9 & Runden 8192 & 65557  \\
			\hline	
		\end{tabular}
		\label{theap1}
\end{table*}

Messung While-Loops in Zyklen:
\begin{table*}[htb]
	\centering
		\begin{tabular}{|l|l|l|}
			\hline	
			Loop 0 & Runden 16 &  171  \\
			\hline	
			Loop 1 & Runden 32 & 301 \\
			\hline	
			Loop 2 & Runden 64 &  535 \\
			\hline	
			Loop 3 & Runden 128 &  1047  \\
			\hline	
			Loop 4 & Runden 264 &  2135  \\
			\hline	
			Loop 5 & Runden 512 &  4119  \\
			\hline	
			Loop 6 & Runden 1024 &  8215  \\
			\hline	
			Loop 7 & Runden 2048 &  16407  \\
			\hline	
			Loop 8 & Runden 4096 &  32791  \\
			\hline	
			Loop 9 & Runden 8192 & 65559  \\
			\hline	
		\end{tabular}
		\label{theap1}
\end{table*}
\\Overhead Timer-Messung: $ 6 Zyklen $
\\Overhead Registerzuweisung: $ 19 - 6 = 13 Zyklen = 19 ns $
\\Messung der Registerzuweisung mit wmb()\footnote{wmb(These functions insert hardware memory barriers in the compiled instruction flow; their actual instantiation is platform dependent. An rmb (read memory barrier) guarantees that any reads appearing before the barrier are completed prior to the execution of any subsequent read. wmb guarantees ordering in write operations, and the mbinstruction guarantees both. Each of these functions is a superset of barrier). Das bedeutet, dass die Schreiboperationen auf die Hardware bis zu dieser Barriere abgeschlossen sein m�ssen und man davon ausgehen kann, dass der Hardwarezugriff bereits erfolgt ist. Somit kann man messen, wie lange ein Hardwarezugriff zum Beschreiben einer LED dauert und diese Zeit von der Gesamtzeit abziehen. Um ein sinnvolles Ergebnis zu erhalten, sollte auch die Messung mit wmb() arbeiten. } 

\subsection{Unterbrechung von Task durch ISR}
Ein Task l�uft und speichert in einer While-Schleife immer die aktuelle Zeit. Er wird durch eine ISR unterbrochen. Sobald der Task wieder anl�uft, wird wieder die Zeit gemessen. Die gesuchte Zeit ist die Different aus Start- und Endzeit.
\\\\Die �bliche Methode f�r Interrupt Service Routinen in FreeRTOS ist, einen hochpriorisierten Task laufen zu lassen, der an einem bin�ren Semaphor blockiert. Wird die ISR aufgerufen, wird dieser Task wieder deblockiert. 
 

\section{RT-Features}
	\begin{enumerate}
		\item Welche Unterschiede/Gemeinsamkeiten gibt es zwischen FreeRTOS und Linux?
		\item Priorit�ten 
		%\item Semaphore
		%\item Message Passing (s. Semaphore)
		%\item Queues (s. Semaphore)
		\item Flags (s. Semaphore)
		\item Posix-Features in Linux
	\end{enumerate}
	
\subsection{RT-Features von Linux/Posix}
\subsubsection{Threads}
Threads in Unix sind Teile von Prozessen. Allerdings hat ein Thread einen eigenen Stack Pointer, eigene Register, Scheduling Properties, Signale und andere Daten. Ein Thread existiert, solange der Elternprozess existiert. Ein Prozess kann mehrere Threads haben. Es kann Datenaustausch von Threads im Rahmen eines Prozesses geben. Ein Thread verbraucht deutlich weniger Ressourcen als ein Prozess. Inter-Thread-Kommunikation ist deutlich schneller, weil alles in einem Adressraum stattfindet. Bei Inter-Prozess-Kommunikation ist mindestens ein Kopiervorgang von Prozess zu Prozess erforderlich.  
\subsubsection{Mutexes}
An Mutexen kann geblockt werden, aber es kann auch ausprobiert werden, ob sie bereits gelockt sind, und dann kann was anderes gemacht werden. 
\subsubsection{Conditions}
Conditions werden im Zusammenhang mit Mutexen benutzt und dienen zur Synchronisation von mehreren Threads, die Datenabh�ngig sind. Zur Benutzung: An einer Condition kann gewartet werden. Zuvor muss ein bestimmter Mutex genommen worden sein. Wenn man den Befehl \textit{pthread\_cond\_wait} ausf�hrt, dann wird damit gewartet und der Mutex automatisch losgelassen. Ein anderer Thread kann sich denselben Mutex holen und dann eine bestimmte Datenverarbeitung an einer Variable durchf�hren. Wenn dadurch die Bedingung erf�llt wird, ruft dieser Thread die Funktion \textit{pthread\_cond\_signal}, die den anderen Thread aufweckt, sobald der Mutex losgelassen wurde. 
\subsubsection{Join}
Threads k�nnen \textit{gejoint} werden. Dieses ist ein Synchronisationsmechanismus von PThreads. Wird  \textit{pthread\_join} aufgerufen, blockiert der aktuelle Thread, bis der zu synchronisierende Thread beendet ist. 
\subsubsection{Message queues}
\textit{mqd\_t}. Eine Queue muss erst erstellt werden. \textit{mq\_send} und \textit{mq\_receive} k�nnen blockend und nicht blockend aufgerufen werden, indem das Flag \textit{O\_NONBLOCK} gesetzt wird. Tasks k�nnen au�erdem dar�ber informiert werden, dass eine Message in der Queue abgelegt wurde. Dieses funktioniert �ber \textit{mq\_notify}. Dar�ber kann ein Handler installiert werden, der ausgef�hrt wird, wenn eine neue Nachricht empfangen wird. Wenn ein anderer Task mit \textit{mq\_receive} an der Queue blockiert, dann wird kein Signal verschickt und der Handler nicht ausgel�st. 
\subsubsection{Scheduling}
\textit{schedPxLib}
\textit{sched\_getScheduler}, gibt entweder SCHED\_FIFO oder SCHED\_RR zur�ck.
\textit{sched\_get\_priority\_max}
\textit{sched\_get\_priority\_max}
sched\_rr\_get\_interval
\subsubsection{Semaphores}
\textit{semPxLib}. Posix Semaphores sind z�hlende Semaphore. Die unterst�tzen Funktionen sind Priorit�tsvererbung, rekurive Semaphore, Timeouts, ... . Posix Mutexes und Condition variables wurden implementiert, indem standardm��ige Semaphore verwendet wurden. 

	
\subsection{RT-Features von FreeRTOS}
Relevante Features:
\subsubsection{Tasks}
Tasks unter FreeRTOS k�nnen mit verschiedenen Priorit�ten erstellt werden. Der idleTask hat immer die niedrigste Priorit�t. Tasks haben verschiedene Zust�nde:
\begin{itemize}
	\item Running: Task wird ausgef�hrt. Es kann zur Zeit nur einen einzigen Task geben, der gerade ausgef�hrt wird.  
	\item Ready: Wartet darauf, ausgef�hrt zu werden, da ein anderer Task gerade vom Scheduler gescheduled wurde
	\item Blocked: Task wartet auf ein Ereignis und wird nicht ausgef�hrt. Grund kann ein Delay oder das Warten an einer Queue oder einem Semaphor sein.
	\item Suspended: Ein Task wurde von einem anderen Task oder sich selber suspendiert. Dieser Status kann nur durch einen Aufruf der Funktion \textit{xTaskResume} fortgesetzt werden. 
\end{itemize}

\subsubsection{Scheduling Policy}
Die Policy wird \textit{Fixed Priority Preemptive Scheduling} genannt. Jedem Task wird eine eigene Priorit�t zugewiesen, wobei Tasks auch die gleiche Priorit�t haben k�nnen. F�r jede Priorit�t existiert eine eigene Liste. 


\subsubsection{Queues}
Queues werden benutzt, um Nachrichten zwischen Tasks auszutauschen. Von einer Queue kann zerst�rend oder nicht zerst�rend gelesen werden und drauf geschrieben werden. Wenn ein Task an einer Queue wartet, kann er f�r eine bestimmte Zeit blockiert werde. Eine Queue kann mit unterschiedlichen Gr��en erzeugt werden. Es gibt auch sogenannte Queue-Sets, die es erm�glichen an mehreren Queue oder auch Semaphoren zu warten.

\subsubsection{Semaphore}
Es gibt drei verschiedene Arten von Semaphoren (s. \ref{semaphore_shuffling_time}), Mutexe, bin�re Semaphore und Counting Semaphors. Semaphore werden als Queue implementiert. Werden Semaphore als Mutexe verwendet, ist die Priorit�tsvererbung ebenfalls verf�gbar. Ein Task kann immer nur an einem Mutex warten, da die Implementierung der Mutexe relativ simpel ist. Wenn ein Mutex von einem niederprioren Task losgelassen wird, kriegt er automatisch seine urspr�ngliche Priorit�t zur�ck. Die zur Priorit�tsvererbung geh�rigen Code-Teile werden �ber Pr�prozessormacros eingebunden. Sollten also keine Mutexe verwendet werden, sollte das Macro auf jeden Fall auf undefiniert bleiben.
	
	
	
\subsection{Task Switching}
Unter Task Switching versteht man die Zeit, die der Scheduler braucht, um von einem Task zu einem anderen zu wechseln. Dieser Wechsel wird nach der Scheduling-Strategie des Schedulers vollzogen, d.h. der Task wird nicht etwa durch einen Interrupt oder durch einen h�her prioren Task unterbrochen.
\subsubsection{FreeRTOS}
\paragraph{Variante 1}
Es werden zwei Tasks \textit{Task1} und \textit{Task2} erzeugt. Diese Tasks haben einen Workload, der darin besteht, in einer For-Schleife eine Variable hoch zu z�hlen. Nach jedem Inkrementieren der Variable wird ein Context-Switch erzwungen (mit taskYIELD()). Wenn die Variable eine bestimmte H�he erreicht hat, wird das Experiment beendet. Es wird dabei die Zeit gemessen, die zwischen dem Betreten des ersten Tasks und dem Verlassen des letzten Tasks vergeht. Ein Task-Switch trifft also zwei Mal so h�ufig auf, wie die Schleife durchgelaufen wird. 
\\\\Von der gemessenen Zeit muss noch der eigentliche Workload abgezogen werden. Daf�r werden vor dem Starten der Tasks zwei For-Schleifen durchlaufen mit der gleichen Anzahl an Durchg�ngen wie in den Tasks.
\paragraph{Variante 2}
Es werden zwei Tasks erzeugt, in denen eine For-Schleife mit der Anzahl der Testdurchl�ufe ausgef�hrt wird. In der Schleife befindet sich in der Reihenfolge:
\begin{itemize}
	\item Starte Messung
	\item Erzwinge Task-Switch mit taskYield()
	\item Stoppe Messung
\end{itemize}
Der Vorteil an dieser Methode ist, dass es keinen Overhead gibt. Diese Messung ist also genauer. Von dem Ergebnis muss noch die Messzeit von sechs Zyklen abgezogen werden. Zu beachten ist, dass beide Tasks damit beginnen, die Startzeit der Messung zu speichern. Das f�hrt dazu, dass die erste Startzeit �berschrieben wird. Die letzte Startzeit ist daf�r ung�ltig und darf nicht in dem Endergebnis ber�cksichtigt werden. 

\subsection{Preemption-Zeit}
<<<<<<< HEAD
Die Preemption-Zeit ist die Zeit, die ben�tigt wird, um einen Task-Switch zu vollziehen, wenn ein niederpriorer Task durch einen Interrupt oder durch einen h�herpriorisierten Task oder einen Interrupt unterbrochen wird. Das bedeutet, der Scheduler wird au�erhalb des regul�ren Tick-Interrupts aufgerufen.

=======
Die Preemption-Zeit ist die Zeit, die ben�tigt wird, um einen Task-Switch zu vollziehen, wenn ein niederpriorer Task durch einen Interrupt oder durch einen h�her priorisierten Task oder einen Interrupt unterbrochen wird. Das bedeutet, der Scheduler wird au�erhalb des regul�ren Tick-Interrupts aufgerufen.
>>>>>>> a4fadd5671de61751ce4fe354d86c6eca5cdf7cd
\subsubsection{FreeRTOS}
Ein niederpriorer Task verrichtet Arbeit. Dieser Task wird nach einer bestimmten Zeit von einem h�her  priorisierten Task unterbrochen. Es gibt zwei Funktionen in FreeRTOS, um einen Delay herbeizuf�hren: \textit{vTaskDelay} und \textit{vTaskDelayUntil}. \textit{vTaskDelay} wacht nach einer bestimmten Anzahl von Ticks auf. Um die Zeit zu messen, kann die Startzeit in einer Endlosschleife im arbeitenden Task dauerhaft ausgelesen werden. Wenn der Task unterbrochen wird, wurde die aktuellste Zeit vorher gespeichert. Sobald der h�her priorisierte Task aufgewacht ist, wird wieder die Zeit gemessen. Das \textit{vTaskDelayUntil} wird nur jeden Tick ausgef�hrt. 
\\\\Eine andere M�glichkeit, die Preemption Zeit zu messen, ist, dass ein hochpriorer Task sich selbst verabschiedet und hinterher von einem niederprioren Task aufgeweckt wird. Dieses f�hrt direkt zu einem Context-Switch.
\\\\Noch eine M�glichkeit ist es, einen hoch priorisierten Task 1 zu suspendieren und dann einen niedriger priorisieren Task 2 laufen zu lassen. W�hrend der Task 2 l�uft, wird ein Interrupt ausgel�st, der Task 1 fortsetzt und somit einen Context-Switch erzeugt.
\subsubsection{Linux}

<<<<<<< HEAD
\subsection{Semaphor Shuffle Time}
Die Semaphor Shuffle Time ist die Zeit, die ein Task braucht, um an einem von einem anderen Task genommenem Semaphor aufzuwachen, wenn dieser wieder losgelassen wird. 
\subsubsection{FreeRTOS}
FreeRTOS hat mehrere Semaphorarten: 
\begin{itemize}
	\item Mutex mit Priority inheritance
	\item Bin�re Semaphore ohne Priority inheritance
	\item Semaphore, die hochgez�hlt werden 
\end{itemize}

Mutexe und Counting Semaphores l�sen beim freigeben des Semaphores einen \textit{portYield} aus, wodurch es zu einer Verz�gerung kommt. 

\paragraph{Mutex}
Ein Mutex wird verwendet, um Mehrfachzugriffe auf Resourcen zu vermeiden. Mutexe k�nnen mit Priority Inheritance verwendet werden, dieser Versuch wird aber mit zwei Tasks gleicher Priorit�ten durchgef�hrt. Wenn ein Task einen Mutex genommen hat, muss dieser ihn auch wieder freigeben. Wenn ein anderer Task an diesem Mutex wartet, wird dieser durch die Freigabe wieder aktiviert.
\\\\Der Versuch kann durchgef�hrt werden, indem ein Task einen Semaphor nimmt und danach ein Context Switch durchgef�hrt wird. Ein zweiter Task versucht ebenfalls den Semaphor zu nehmen, wird aber blockiert. Als Folge erfolgt wieder ein Task Switch zur�ck zum ersten Task. Dieser startet eine Zeitmessung, l�sst den Semaphor wieder los und veranlasst einen Context Switch. Nun wacht Task 2 auf, da der Semaphor losgelassen wurde. Die Zeitmessung wird beendet. Der Task l�sst den Semaphor wieder los und mit einem Context Switch beginnt ein weiterer Durchlauf des Versuchs.
\\\\Im Endergebnis muss ber�cksichtigt werden, dass ein Mal \textit{xSemaphoreTake}, ein Mal \textit{xSemaphoreGive} und ein Context Switch von der Gesamtzeit abgezogen werden.

\paragraph{Bin�re Semaphore}
Bin�re Semaphore in FreeRTOS werden �hnlich wie Signale verwendet und diesen eher der Synchronisation als dem Vermeiden von Mehrfachzugriffen. 
\\\\In diesem Versuch wartet der erste Task an dem Semaphor. Ein zweiter Task wird gestartet, initiiert eine Zeitmessung, gibt den Semaphor frei und macht einen Context Switch. Durch das Freigeben ist nun der ander Task wieder aufgewacht und beendet die Zeitmessung. 
\\\\Von dem Endergebnis muss ein Mal \textit{xSemaphoreGive} abgezogen werden. 

\paragraph{Counting Semaphores}
Counting Semaphores sind Semaphore, die ein bestimmtes Kontingent haben und die hochgez�hlt werden, wenn ein Semaphor freigelassen wird (also eine Ressource verf�gbar ist) und wieder runtergez�hlt werden, wenn ein Semaphor genommen wird (also eine Ressource belegt ist). Ist keine Ressource verf�gbar, wird an dem Semaphor gewartet. Ein Task wird wieder aktiv, sobald eine Ressource verf�gbar wird. 
\\\\Diese Semaphore k�nnen �hnlich wie Mutexe vermessen werden. Da der Mutex zuerst genommen wird, muss der Semaphor mit seiner maximalen Anzahl (im Versuch 1) initialisiert werden. Der Rest des Versuchs ist analog zum Mutex-Versuch. Am Ende m�ssen auch die gleichen Werte abgezogen werden. 

	
\subsubsection{Linux}

\subsection{Deadlock breaking time}
Diese Zeit ist die Zeit, die ben�tigt wird, um einen Deadlock durch Priorit�tsinversion wieder aufzul�sen. G�be es diese nicht, w�re folgende Situation ein Deadlock:
\\\\Task 1 hat die niedrigste Priorit�t und nimmt sich Mutex M. Bevor M freigegeben wird, wird Task 1 durch Task 2 mit einer h�heren Priorit�t unterbrochen. Task 2 wiederum wird durch Task 3 unterbrochen, was die h�chste Priorit�t hat. Task 3 greift nach Mutex M und wird blockiert. Ohne Priorit�tsinversion w�rde Task 2 weiterlaufen und Task 1 f�r immer unterbrechen, sodass die f�r Task 3 ben�tigte Ressource nie freigegeben wird. Priorit�tsinversion sorgt daf�r, dass Task 1 vor�bergehend die Priorit�t von Task 3 bekommt, nicht mehr von Task 2 blockiert wird, und somit den Mutex wieder freigeben kann. Somit kann der Task mit der h�chsten Priorit�t, Task 3, seine Arbeit beenden.
\\\\F�r die Messung ist es irrelevant, ob Task 1 zwischen der Priorit�tsinversion und dem Zur�ckgeben des Mutex noch Arbeit verrichtet. Um die Messung so genau wie m�glich zu halten, wird der Mutex direkt zur�ckgegeben.

\subsubsection{FreeRTOS}

\subsubsection{Linux}

\subsection{Message Passing Latency}
Die Message Passing Latency ist die Zeit, die ein Task braucht, um eine Message von einem anderen Task zu empfangen. In FreeRTOS werden diese Messages �ber Queues transportiert. Die Zeit messen kann mit folgendem Aufbau:
\\\\Task 1 wird gestartet und wartet an der Queue auf eine Nachricht. Dabei wird der Task blockiert. Dann startet Task 2 und schreibt eine Nachricht in die Queue. Dieses bewirkt, dass der erste Task deblockiert wird. Die Zeitmessung beginnt vor dem Versenden der ersten Nachricht und endet mit dem Empfang. 
	
\subsubsection{FreeRTOS}

\subsubsection{Linux}	
=======
\subsection{Semaphor Shuffle Time}\label{semaphore_shuffling_time}
Die Semaphor Shuffle Time ist die Zeit, die ein Task braucht, um an einem von einem anderen Task genommenem Semaphor aufzuwachen, wenn dieser wieder losgelassen wird. Dieser Versuch kann durchgef�hrt werden, indem ein Task einen Semaphor nimmt und danach ein Context-Switch durchgef�hrt wird.

\subsection{Message Passing}
Task 1 wartet auf eine Nachricht, Task 2 schickt eine Nachricht, Task 1 wacht auf und empf�ngt die Nachricht.
\subsubsection{FreeRTOS}
Messages werden durch Queues implementiert.
\subsubsection{Linux}
>>>>>>> a4fadd5671de61751ce4fe354d86c6eca5cdf7cd
	
\subsection{DLB Time}
Die Deadlock breaking time ist die Zeit, die ben�tigt wird, um einen potenziellen Deadlock durch Priorit�tsvererbung aufzul�sen. Folgendes Beispiel der Priorit�tsinverion verdeutlicht die Wichtigkeit der Priorit�tsvererbung:
Ein niederpriorer Task greift nach einem Mutex M. Dann wird er von einem mittelprioren Task unterbrochen. Dieser wiederum wird von einem hochprioren Task unterbrochen, welcher ebenfalls nach dem Mutex M greift und blockiert. Damit wird wieder Task 2 aufgerufen, welcher nun f�r immer Task 1 und damit auch den Mutex M blockiert, sodass der h�her priorisierte Task niemals ausgef�hrt wird.
\\\\Mit der Priorit�tsvererbung bekommt Task 1 vor�bergehend die Priorit�t von Task 3. Damit kann er nicht von Task 2 blockiert werden und nach getaner Arbeit den Mutex wieder freigeben. Damit kann der hoch priore Task seine Arbeit verrichten.

\subsubsection{FreeRTOS}
Es werden zwei Tasks erzeugt: Task 1 und Task 3. Task 3 ist der hochpriore Task und legt sich direkt schlafen. Task eins l�uft in einer Schleife. Zuerst nimmt er sich den Mutex, dann startet er Task 2. Task 2 hat die mittlere Priorit�t und setzt Task 3 fort. Task 3 versucht nun nach dem Mutex zu greifen und wird blockiert. Da f�r Mutexe in FreeRTOS eine Priorit�tsvererbung stattfindet, wird wie erwartet Task 1 fortgesetzt und gibt den Mutex wieder frei. Es wird die Zeit gemessen, ab der Task 3 versucht, auf den Mutex zuzugreifen, bis der Mutex freigegeben und der Task deblockiert wird. 

\subsubsection{Linux}  	
 	
\section{Speicherzugriffe}
	\begin{enumerate}		
		\item Speicherplatzverbrauch des gesamten Systems
		\item MPU-Unterst�tzung
		\item In welchem Rahmen sind dynamische Speicherzugriffe m�glich?
		\item Ggf. Zeitverbrauch bei Speicherallokation/-fragmentierung
			\begin{enumerate}		
				\item Allokation von z.B. 1000 Paketen und Messen der Zeit
				\item Vergleich von Context Switch mit Speicher Allokation und ohne (?)
				\item Vergleich von verschiedenen Methoden der Speicherallokation $ \rightarrow $ Was ist der Worst Case, der passieren kann?
			\end{enumerate}
	\end{enumerate}
	
\subsection{FreeRTOS}
Es gibt immer eine Mindestfrakturgr��e. Au�erdem sind die Funktionen Thread-Save, d.h. k�nnen durch keinen anderen Task unterbrochen werden. Ausnahmen davon bildet je nach Implementierung Fall 3.

\subsubsection{Heap\_1.c}
Bl�cke werden allokiert, wenn genug Speicher da ist und nie wieder freigegeben.

\begin{table*}[htb]
	\centering
		\begin{tabular}{|l|p{11cm}|}
			\hline
			 Zugriffszeit & Konstant \\	
			\hline
			 Worst Case & Nicht mehr gen�gend Speicher vorhanden  \\
			\hline
			 Schlussfolgerung & Schnell, aber vorher �berlegen, ob der Speicher f�r die Lebensdauer der Anwendung reicht. \\
			\hline
			 Testfall & Einfaches Allozieren, da kein Rechenaufwand durch Freigaben notwendig. \\
			\hline	
		\end{tabular}
		\label{theap1}
\end{table*}


\subsubsection{Heap\_2.c}
Es gibt eine minimale Blockgr��e. Es gibt eine Liste, in der die Bl�cke nach Gr��e sortiert sind. Es wird immer der n�chst gr��te Block alloziert $ \rightarrow $Iteration durch Liste. Kein Verschmelzen von Blocks bei Freigabe. Zu gro�e Blocks werden aufgeteilt. Der neu entstandene Block wird wieder in die Liste einsortiert. Nur sinnvoll, wenn der allozierte Speicher immer in etwa die gleiche Gr��e hat. 

\begin{table*}[htb]
	\centering
		\begin{tabular}{|l|p{11cm}|}
			\hline
			 Zugriffszeit & Am Anfang konstant, weil nur ein Block. Sobald die Liste mehrere Elemente besitzt, ist die Zugriffszeit linear abh�ngig von der L�nge der Liste. \\	
			\hline
			 Worst Case & Nicht mehr gen�gend Speicher vorhanden oder es ist Speicher vorhanden, aber nicht mehr an einem St�ck oder es gibt sehr viele kleine Segmente in der Liste und nur ein gr��eres ganz hinten  \\
			\hline
			 Schlussfolgerung & Durch Freigaben langsamer als in Fall eins. Nicht sinnvoll, wenn allozierte Blockgr��e variiert.\\
			\hline
			 Testfall &  Allozieren von m�glichst vielen minimal gro�en Bl�cken und einem, der die doppelte Gr��e hat. Alle wieder freigeben $ \rightarrow $ Lange Liste mit vielen Eintr�gen $ \rightarrow $ Nochmal den gr��eren Block allozieren. Die Zeit f�r die L�ngste Freigabe kann auch gemessen werden. \\
			\hline	
		\end{tabular}
		\label{theap2}
\end{table*}

\subsubsection{Heap\_3.c}
Maskierte malloc und free Aufrufe des jeweiligen Compilers.

\begin{table*}[htb]
	\centering
		\begin{tabular}{|l|p{11cm}|}
			\hline
			 Zugriffszeit &  \\	
			\hline
			 Worst Case &  \\
			\hline
			 Schlussfolgerung & \\
			\hline
			 Testfall &  \\
			\hline	
		\end{tabular}
		\label{theap3}
\end{table*}

\subsubsection{Heap\_4.c}
Liste mit Blockzeigern und Blockgr��e. Liste wird durchsucht, bis ein passendes Element gefunden wird. Bei Freigabe werden nebeneinander liegende Bl�cke wieder zusammengef�hrt.

\begin{table*}[htb]
	\centering
		\begin{tabular}{|l|p{11cm}|}
			\hline
			 Zugriffszeit & Wie in Fall zwei, aber insgesamt schneller, da Bl�cke bei der Freigabe wieder zusammengef�hrt werden und insgesamt tendenziell weniger Bl�cke durchiteriert werden m�ssen.\\	
			\hline
			 Worst Case & Wie in Fall zwei\\
			\hline
			 Schlussfolgerung & Flexibelste Alternative, Freigabe ist geringf�gig langsamer als in Fall zwei, weil Bl�cke noch zusammengef�hrt werden.\\
			\hline
			 Testfall &  Allozieren wie in Fall zwei. Freigabe von jedem zweiten Block, sodass Speicher segmentiert bleibt. Dann nochmal den hintersten Block allozieren. \\
			\hline	
		\end{tabular}
		\label{theap4}
\end{table*}

\subsection{Verifizierung}
\begin{itemize}
	\item Unter welchen Voraussetzungen ist eine Verifizierung m�glich?
	\item Verifizierung bei einem ganz bestimmten Szenario
\end{itemize}

\subsection{Multiprozessorunterst�tzung}
\chapter{Measurements}\label{ch_measurements}
This chapter describes the measurements performed for benchmarking the \acp{OS} FreeRTOS and LinuxRT.
First, the conditions under which the experiments have taken place are described.
This includes hardware platform, time measuring technique and setup of the operation systems under test.
After that every measurement is described in detail beginning with the boot time.

\section{Test Environment and tools}
This section contains a description of the test environment - the hardware platform, the development tools and the configuration of the used \acp{OS}.
\subsection{Hardware Platform}
The underlying hardware platform is a Xilinx ZC702 Evaluation Board \cite{xilinx:zc702_ev_board} with an Zynq-7000 XC7Z020 \cite{xilinx:zynq7000}.
The XC7Z020 chip integrates a \ac{PS} and a \ac{PL} on a single die.
Two ARM Cortex-A9 MPCore application processors running at 666 MHz are located in the \ac{PS}.
Both cores inherit a 32 KB instruction and a 32 KB data level 1 cache.
Moreover, they share a 512 KB level 2 cache and 256 K SRAM.
The \ac{PL} is an Artix-7 from the Xilinx's 7 series FPGA technology and can be used to implement custom hardware designs.
\par
The \ac{PS} uses so called \ac{MIO} pins to connect to peripheral devices.
Those pins can be configured to connect either \acp{GPIO}, Ethernet, \ac{SPI} and others.
The \ac{PL} and the \ac{PS} can be connected using the \ac{EMIO} pins.
Moreover, the \ac{EMIO} pins can be used to extend the number of connected peripherals.
\par
When a tool for the Zynq platform is developed, most likely both \ac{PS} and \ac{PL} will be included.
Therefore, an interrupt generator was implemented in the  \ac{FPGA} design to utilize the \ac{PL}.
It is connected as a \ac{GPIO}.
\par
%\subsection{Development Tools}
The Xilinx Design Suite version 14.4 was used to create the test environment and the tests.
The \ac{FPGA} part was designed and synthesized in Xilinx \ac{ISE}.
The software was developed using Xilinx \ac{SDK}.

\subsection{\ac{OS} Configuration}
The operation systems under test are FreeRTOS and LinuxRT.
Both need to be configured such that the implemented interrupt generator can be utilized.

\subsubsection{FreeRTOS}
As already meantioned, interrupt management does not effect the kernel in FreeRTOS.
Therefore, the \ac{OS} itself does not need to be configured.
Still, the hardware needs to be initialized appropiately on system startup (refer to \ref{ss_interrupts_in_freertos}). 

\subsubsection{LinuxRT}



\section{Boot time}
\section{Interrupt Latency}
\section{Preemption Time}
\section{Semaphore Shuffling Time}
\section{Message Passing Time}
\section{Deadlock Breaking Time}
\chapter{Results}\label{ch_results}
\chapter{Concluding Remarks}\label{ch_conclusion}
Concluding, the first part of this chapter gives a brief summary of this work.
The second part introduces possibilities of future work and extensions.

\section{Summary and Reflection}   
In the following, the main parts of this thesis are summarized:

\begin{itemize}
	\item 
	The first chapter gives an introduction to the topic. 
	It explains the timing problems in real-time systems and the role and challenges of \acp{RTOS}.
	Moreover, it outlines why the \acp{OS} LinuxRT and FreeRTOS were chosen for the comparison.
	\item
	In the second chapter, a detailed background on operating systems is given with focus on FreeRTOS and LinuxRT.
	Based on that, sources of delays caused by hard ware and the \ac{OS} are analyzed.
	This is expressed by a formula for application latency estimation.
	Finally, it is discussed which delays occur in the \acp{OS} under test and how they can be minimized or completely eliminated.
	\item
	Derived from the analysis in chapter two, a set of benchmarks to quantify the operating systems is defined in chapter three.
	These are mainly based on the Rhealstone real-time benchmark proposed by Kar.
	Some extensions to measure boot time and detailed interrupt handling are made.
	Further, the underlying test hardware platform is introduced.
	\item
	The fourth chapter contains the results of the test which were described in chapter three.
	Further, a comparison and interpretation of these results is given.
	One important result is the maximum interrupt latency under network load.
	Moreover, the complexity of the Linux POSIX \ac{API} is clearly visible in the results.
	This enables concrete optimization of sources which have strong impact on the system behavior.	
	The results obtained give the possibility to compare different synchronization methods, tune the performance of an application and estimate whether an \ac{OS} is suitable for a specific task or not.
	\item 
	The last chapter utilizes the model derived in chapter \ref{ch_background} and the results from chapter 	\ref{ch_results} to actually estimate the latency of two theoretical applications.
\end{itemize}

The goal of this work was to make a comparison between the two operating systems LinuxRT and FreeRTOS. 
While it was the main motivation to get an impression of the LinuxRT performance, FreeRTOS has been used as reference system.
FreeRTOS is a light-weight system with a low memory footprint which is certifiable, very fast and predictive. 
Jitter from the \ac{OS} is barely available. 
The drawback is that any change in an application includes recompilation and reprogramming of the hardware.
Moreover, every driver and its handling has to be implemented manually if needed.
LinuxRT supports lots of drivers and is convenient for application design, but the complexity of the system has a high impact on its real-time capability.
This work uses the Rhealstone method to create a benchmark suite and to obtain the possibilities and limitations of the two \acp{OS}.
With minor modifications, the test programs can also be run on different hardware platforms and used to compare them.
The result of this work is not a concrete answer to the question when to utilize which operating system but a guideline.
The model and the parameters resulting from implementing the Rhealstone benchmark provide a solid base to estimate the \ac{OS} overhead for real-time applications. 
It shows the risks and limitations of LinuxRT but also its possibilities. 

\section{Outlook and future Work}
The benchmarks provide a solid base to compare operating systems.
Yet, there are always possibilities to enhance and extend the functionality.

\subsection{Extension of Model Parameters}
To make the estimation technique available for more applications, the number of parameters for the model has to be increased. 
One important parameter is for example the time which it takes to access hardware from an application, for instance to turn on an \ac{LED}. 
Although this time is part of the interrupt latency, some applications might need to write to devices without the occurrence of an interrupt.
Another factor is the time until the interrupt reaches the application without producing an output. 

\subsection{System Setup}
As already mentioned (see chapter \ref{s_summary}), the system setup used in this work can be changed.
One special case to consider is the multicore support of both systems.
A big drawback of FreeRTOS is that there does not exist an official version which supports multiple \acp{CPU}.
Nevertheless, some effort has been made in this domain \cite{mistry:affmaer}.
As Linux supports this feature in its kernel configuration, it can be exploited during system design.
For example, non-critical interrupts can be handled by only one specific \ac{CPU} and the real-time tasks can run on different one.
This is often done when the underlying hardware resources are available. 
\par
Further, especially the test programs for Linux can be run on different hardware without large modifications.
Consequently, they can not only be used to compare \acp{OS} to each other but also hardware platforms.

\subsection{Periodic Timers}
One important aspect for real-time system which was not handled in this work is the implementation of periodic tasks.
As the name suggests, these tasks are scheduled periodically by the operating system.
Depending on their function, some tasks may need different periods to execute their work.
Therefore, it is important to investigate whether the \ac{OS} provides a timer function and its accuracy.
Currently, both FreeRTOS and LinuxRT provide a timer function.
In FreeRTOS, the time resolution depends on the tick interval of the system.
The Linux kernel provides high-resolution timers independent of the timer tick.

\subsection{Power Consumption}
Power consumption may not obviously seem to be related to the operating system but more to the underlying hardware platform.
Still, the \ac{OS} can provide functions to control this hardware and, for instance, put a device into sleep mode or to reduce the \ac{CPU} frequency.
Especially Linux provides a lot of kernel configurations to enable and disable power saving modes.
Another way to save power is to put the kernel into a tickless mode (refer to section \ref{ss_timer_tick} for details).
This function is available in both operating systems.
Going one step further, the Linux developers have released kernel 3.10 which provides a first version of a (nearly) full tickless mode \cite{corbet:nftoi}.
This means not only stopping the timer tick when the system is idle but removing the tick almost completely. 
Doing this is a big step towards saving power and as well a huge step towards removing the largest component of $t_{jitter}$ from the system.

\subsection{Light-weight APIs}
An obvious result from the experiments was the inefficiency of the POSIX \ac{API} compared to the FreeRTOS \ac{API}.
This is not only related to more nested function calls but also to the amount of configurations this \ac{API} provides.
It is worth investigating how much of this functionality is really necessary.
Linux' performance is up to nine times slower for \ac{API} calls than the FreeRTOS one, six times if the application can be considered free of priority inversion.
A solution to this problem is the implementation of a light-weight \ac{API} for real-time applications with less configuration options.
Consequently, the operating system would become more attractive in terms of performance compared to its competitors.




% --------------------------------------------------------------------
% Anhang
% --------------------------------------------------------------------
\clearpage
\begin{appendix}

%\appendix
\markboth{Appendix}{Appendix}

\chapter*{Appendix}\addcontentsline{toc}{chapter}{Appendix}
\section*{\RM{1}  \space Contents on CD-ROM}
\subsection*{\RM{1}A  \space This document in digital Form}
\subsection*{\RM{1}B  \space Hardware Platform}
\subsection*{\RM{1}C  \space LinuxRT Configuration Files}
\subsection*{\RM{1}D  \space Rhealstone Benchmark Software}


\end{appendix}

% --------------------------------------------------------------------
% Index
% --------------------------------------------------------------------
\clearpage
\printindex

% --------------------------------------------------------------------
% Literaturverzeichnis
% --------------------------------------------------------------------
% Nicht direkt referenzierte, aber benutzte Literatur
%\nocite{wasauchimmer}
%
\clearpage
%\bibliographystyle{geralpha}
\bibliographystyle{gerplain}
\bibliography{literatur} % welche bib-Dateien

% --------------------------------------------------------------------
% Dokument-Ende
% --------------------------------------------------------------------
\end{document}
%%%%%%%%%%%%%%%%%%%%%%%%%%%%%%
